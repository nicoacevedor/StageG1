% Template:     Template Reporte LaTeX
% Documento:    Archivo principal
% Versión:      2.1.4 (05/01/2021)
% Codificación: UTF-8
%
% Autor: Pablo Pizarro R.
%        Facultad de Ciencias Físicas y Matemáticas
%        Universidad de Chile
%        pablo@ppizarror.com
%
% Sitio web:    [https://latex.ppizarror.com/reporte]
% Licencia MIT: [https://opensource.org/licenses/MIT]

% CREACIÓN DEL DOCUMENTO
\documentclass[letterpaper,oneside]{article}

% INFORMACIÓN DEL DOCUMENTO
\def\titulodelreporte {Rapport de Stage}
\def\temaatratar {Stage G1}
\def\fechadelreporte {26 février 2021}

\def\autordeldocumento {Nicolas Acevedo}
\def\nombredelcurso {Stage G1}
\def\codigodelcurso {}

\def\nombreuniversidad {Universidad de Chile}
\def\nombrefacultad {Facultad de Ciencias Físicas y Matemáticas}
\def\departamentouniversidad {Departamento de Ingeniería Matemática}
\def\imagendepartamento {departamentos/dim}
\def\localizacionuniversidad {Santiago, Chile}

% IMPORTACIÓN DEL TEMPLATE
\input{template}

% INICIO DE PÁGINAS
\begin{document}
	
% CONFIGURACIÓN DE PÁGINA Y ENCABEZADOS
\templatePagecfg

% CONFIGURACIONES FINALES
\templateFinalcfg

% ======================= INICIO DEL DOCUMENTO =======================

% Título y nombre del autor
\inserttitle

% Resumen o Abstract

% Ejemplo, se puede borrar
\sectionanum{Introduction}
De nos jours, il est impératif de respecter les mesures de sécurité sociale pour éviter la contagion et la propagation du virus COVID-19, comme le lavage constant des mains, respecter la distance sociale d'au moins un mettre entre les personnes, et l'utilisation du masque pour se protéger et protéger les autres.\\

Avec l'objectif d'aider à respecter ces mesures, un système robotique a été développé pour détecter et signaler si les personnes portent ses masques et gardent la distance minimale afin de diminuer la probabilité de propagation du virus.

\sectionanum{Intelligence Artificiel actuellement}

%\input{example}

% FIN DEL DOCUMENTO
\end{document}
